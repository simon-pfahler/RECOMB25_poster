%!TeX program = LuaLaTeX

\DocumentMetadata{lang=en-US}
\documentclass[tikz]{standalone}

\usepackage{colorblind}
\usepackage{amsmath}
\usepackage{qrcode}
\usepackage{wasysym}

\usepackage{cmbright}
\usepackage[sfdefault]{josefin}
\usepackage{fontspec}

\newfontfamily{\handwritten}{augie}
\DeclareTextFontCommand{\texthandwritten}{\handwritten}

\newfontfamily{\firatt}{FiraCodeNerdFont-Regular}
\DeclareTextFontCommand{\texttt}{\firatt}

\newfontfamily{\firabtt}{FiraCodeNerdFont-SemiBold}
\DeclareTextFontCommand{\textbtt}{\firabtt}

\usepackage{anyfontsize}

\usepackage{xcolor}

\usepackage{tikz}
\usetikzlibrary{arrows.meta}
\usetikzlibrary{backgrounds}
\usetikzlibrary{bending}
\usetikzlibrary{calc}
\usetikzlibrary{decorations.pathmorphing}

\usepackage{pgfplots, pgfplotstable}
\pgfplotsset{compat=1.18}

\usepackage[sorting=none, style=nature]{biblatex}
\addbibresource{references.bib}

% >>> do not print titles
\renewbibmacro*{title}{}
% <<< do not print titles

\usepackage{hyperref}

\renewcommand\vec[1]{\mathbf{#1}}
\renewcommand{\labelitemii}{\scalebox{0.7}{$\circ$}}

\def\w{84.1}
\def\h{118.9}
\newlength\wl\setlength\wl{84.1cm}
\newlength\hl\setlength\hl{118.9cm}

\def\titlesize{\fontsize{100}{150}\selectfont}
\def\headingsize{\fontsize{60}{90}\selectfont}
\def\smallheadingsize{\fontsize{50}{75}\selectfont}
\def\normalsize{\fontsize{36}{45}\selectfont}
\def\refsize{\fontsize{30}{45}\selectfont}
\def\reffsize{\fontsize{30}{35}\selectfont}
\def\commentsize{\fontsize{20}{22}\selectfont}

\definecolor{BG}{RGB}{252,234,222}
\definecolor{HL}{RGB}{255,138,91}
\definecolor{Block}{RGB}{213,233,245}
\definecolor{BlockBorder}{RGB}{24,143,167}

\def\point(#1,#2){\draw[line width=0.2cm, red] (#1+0.5,#2+0.5) -- ++(-1,-1) ++(1,0) -- ++(-1,1);}
\def\followsarrow{\tikz{
    \draw[-{Stealth[round,scale=1.25]},line width=1mm,line cap=round] (0,0) -- (0.8,0);
}}
\def\straightarrow(#1,#2)[#3,rotate=#4]{%
    \begin{scope}[shift={(#1,#2)}]
        \fill[#3, rounded corners = 0.5cm, shift={(0.1,-0.4)}, rotate=#4] (0,0) -- ++(5,0) -- ++(0,1) -- ++(1.5,-1.5) -- ++(-1.5,-1.5) -- ++(0,1) -- ++(-5,0) -- cycle;
        \draw[line cap=round, line width=2mm, rounded corners = 0.5cm, rotate=#4] (0,0) -- ++(5,0) -- ++(0,1) -- ++(1.5,-1.5) -- ++(-1.5,-1.5) -- ++(0,1) -- ++(-5,0);
    \end{scope}}
\def\curvedarrow(#1,#2)[#3,rotate=#4]{%
    \begin{scope}[shift={(#1,#2)}]
        \fill[#3, rounded corners = 0.5cm, shift={(0.1,-0.4)}, rotate=#4] (0,0) to[bend left=20] ++(5,2) -- ++(0,1) -- ++(1.5,-1.5) -- ++(-1.5,-1.5) -- ++(0,1) to[bend right=20] ++(-4.3,-1.7) -- cycle;
        \draw[line cap=round, line width=2mm, rounded corners=0.5cm, rotate=#4] (0,0) to[bend left=20] ++(5,2) -- ++(0,1) -- ++(1.5,-1.5) -- ++(-1.5,-1.5) -- ++(0,1) to[bend right=20] ++(-4.3,-1.7);
    \end{scope}}
\def\curvedarrowinv(#1,#2)[#3,rotate=#4]{%
    \begin{scope}[shift={(#1,#2)}]
        \fill[#3, rounded corners = 0.5cm, shift={(0.1,-0.4)}, rotate=#4] (0,0) to[bend right=20] ++(5,-2) -- ++(0,-1) -- ++(1.5,1.5) -- ++(-1.5,1.5) -- ++(0,-1) to[bend left=20] ++(-4.3,1.7) -- cycle;
        \draw[line cap=round, line width=2mm, rounded corners=0.5cm, rotate=#4] (0,0) to[bend right=20] ++(5,-2) -- ++(0,-1) -- ++(1.5,1.5) -- ++(-1.5,1.5) -- ++(0,-1) to[bend left=20] ++(-4.3,1.7);
    \end{scope}}

\begin{document}
\normalsize
\begin{tikzpicture}[
	x=1cm,y=1cm,
	line width=1mm,
	background rectangle/.style={fill=BG},
	show background rectangle,
	commentarrow/.style={
	line width=0.07cm,
	-{Stealth[round, scale=1.1]},
	line cap=round
	},
	blockcolors/.style={
			fill=Block,
			draw=BlockBorder,
		},
	]
	\useasboundingbox (-\w/2,-\h/2) rectangle (\w/2,\h/2);

	% >>> different background for biological results
	\fill[BG!50, line width=1cm] (5,-7) ..controls ++(350:10) and ++(80:15)..
	(25,-20) ..controls ++(80:-5) and ++(130:10)..
	(30,-40) --
	(15,-45) ..controls ++(180:10) and ++(150:-10)..
	(-5,-30) ..controls ++(150:10) and ++(90:-20)..
	(-20,-10) ..controls ++(90:10) and ++(20:-10)..
	(-5,0) --
	cycle;
	% <<< different background for biological results

	% >>> roter Faden
	\draw[HL!60, line width=2.5cm] (-12,24) ..controls ++(5:10) and ++(90:8)..
	(13,15) ..controls ++(90:-5) and ++(20:10)..
	(-5,0) ..controls ++(20:-10) and ++(90:10)..
	(-20,-10) ..controls ++(90:-20) and ++(150:10)..
	(-5,-30) ..controls ++(150:-10) and ++(180:10)..
	(15,-45);
	% <<< roter Faden

	% >>> Titelblock
	\fill[HL] (-\w/2,0.295*\h) .. controls ++(0.7*\w,0) and ++(-0.3*\w,0.05*\h).. (\w/2,0.33*\h) -- (\w/2,\h/2) -- (-\w/2,\h/2) -- cycle;
	\draw[HL!40!black,line width=0.5cm] (-\w/2,0.295*\h) .. controls ++(0.7*\w,0) and ++(-0.3*\w,0.05*\h).. (\w/2,0.33*\h);
	\node[anchor=north west] at (-0.48*\w,0.485*\h)
	{
		\parbox[t]{0.95\wl}{
			\titlesize Exploiting weak modularity in cancer progression to infer large Mutual Hazard Networks\\[0.5ex]
			\normalsize\hphantom{m}Simon Pfahler\textsuperscript{1}, Leon Ernstberger\textsuperscript{1}, Peter Georg\textsuperscript{1}, Andreas Lösch\textsuperscript{2}, Rudolf Schill\textsuperscript{3}, Lars Grasedyck\textsuperscript{4}, Rainer Spang\textsuperscript{2}, Tilo Wettig\textsuperscript{1}\\
			\refsize\hphantom{m}\textsuperscript{1} Department of Physics, University of Regensburg, \textsuperscript{2} Faculty of Informatics and Data Science, University of Regensburg,\\
			\hphantom{m}\textsuperscript{3} Department of Biosystems and Engineering, ETH Zürich, \textsuperscript{4} Institute for Geometry and Applied Mathematics, RWTH Aachen
		}
	};
	\node[fill=BG!50, rounded corners=0.2cm,anchor=west] at (-0.495*\w,41) (QR) {\qrcode[height=6.5cm]{https://simon-pfahler.github.io/permanent/RECOMB25/RECOMB25.html}};
	\node[anchor=north, inner sep=0] at (QR.south) {online version};
	\node[anchor=west] at ($(QR.east)+(1,-0.3)$) (UR) {\includegraphics[height=6.5cm]{graphics/ur-logo-bildmarke-grau.eps}};
	\node[anchor=west, fill=BG!50, rounded corners=0.5cm] at ($(UR.east)+(1,0)$) {\includegraphics[height=5cm]{graphics/logo_sdv.pdf}};
	% <<< Titelblock

	% >>> Summary block
	\begin{scope}[shift={(-40,32)}]
		\filldraw[blockcolors,line width=0.3cm]
		(-1,-7) ..controls ++(90:10) and ++(175:12)..
		(20,2) ..controls ++(175:-16) and ++(15:18)..
		(12,-28) ..controls ++(15:-14) and ++(90:-10)..
		cycle;
		\node[anchor=north west] at (0,1)
		{
			\parbox[t]{30cm}{
				\hspace{3cm}\headingsize\textbf{Summary}
				\normalsize
				\begin{itemize}
					\item In cancer progression, the rate of occurence\\of genetic events depends on the state of a\\tumor
					\item Mutual Hazard Networks infer promoting and\\inhibiting effects between genetic events\\from patient data
					\item Splitting the events in a dataset into clusters\\allows us to infer approximate MHNs for\\hundreds of events, overcoming a major\\runtime limitation of MHN
					\item Investigating the obtained clusters can\\give valuable input into the under-\\lying biology, e.g.\ the role that\\different events play in cancer\\
					      \hphantom{mn}progression
				\end{itemize}
			}
		};
	\end{scope}
	% <<< Summary block

	% >>> MHN graph example
	\begin{scope}[shift={(0,30)}]
		\node[circle, draw=violet, fill=violet!20] at (80:4.5) (A) {\headingsize A};
		\node[circle,draw=yellow, fill=yellow!20] at (210:4.5) (B) {\headingsize B};
		\node[circle,draw=green, fill=green!20] at (340:4.5) (C) {\headingsize C};
		\draw[blue, line width=1mm, ->, shorten <=0.2cm, shorten >=0.2cm, line cap=round] (A) to[bend right=25] node[above, sloped, black!50] {\commentsize\texthandwritten{promoting}} (B);
		\draw[red, line width=1.5mm, -|, shorten <=0.2cm, shorten >=0.2cm, line cap=round] (B) to[bend right=25] node[below, sloped, black!50] {\commentsize\texthandwritten{inhibiting}} (C);
		\draw[blue, line width=2mm, ->, shorten <=0.2cm, shorten >=0.2cm, line cap=round] (C) to[bend right=25] (A);%node[above, sloped, black!50] {\commentsize\texthandwritten{stochastic}} (A);
		\draw[black!50, commentarrow, shorten <=0.1cm, shorten >=0.1cm] (-0.3,1.5) arc (110:435:1cm);
		\node[black!50] at (0,-1) {\commentsize\texthandwritten{circular}};
	\end{scope}
	% <<< MHN graph example

	% >>> Markov chain visualization
	\begin{scope}[line cap=rect,shift={(-6,21)}]
		% names
		\node[violet, anchor=south, inner sep=5pt] at (0.5,0) {\refsize A};
		\node[yellow, anchor=south, inner sep=5pt] at (1.5,0) {\refsize B};
		\node[green, anchor=south, inner sep=5pt] at (2.5,0) {\refsize C};

		% first line
		\fill[white] (0,0) rectangle ++(1,-1);
		\fill[white] (1,0) rectangle ++(1,-1);
		\fill[white] (2,0) rectangle ++(1,-1);
		\draw (0,0) -- ++(3,0) -- ++(0,-1) -- ++(-3,0) -- ++(0,1) ++(1,0) -- ++(0,-1) -- ++(1,0) -- ++(0,1);

		% connections
		\draw[line cap=round,black!50,shorten >=1.5cm, shorten <=1.5cm,->] (1.5,-0.5) -- (-2,-3);
		\draw[line cap=round,black!50,shorten >=0.75cm, shorten <=0.85cm,->] (1.5,-0.5) -- (1.5,-3);
		\draw[line cap=round,black!50,shorten >=1.5cm, shorten <=1.5cm,->] (1.5,-0.5) -- (5,-3);

		% second line
		\fill[violet] (-3.5,-2.5) rectangle ++(1,-1);
		\fill[white] (-2.5,-2.5) rectangle ++(1,-1);
		\fill[white] (-1.5,-2.5) rectangle ++(1,-1);
		\fill[white] (0,-2.5) rectangle ++(1,-1);
		\fill[yellow] (1,-2.5) rectangle ++(1,-1);
		\fill[white] (2,-2.5) rectangle ++(1,-1);
		\fill[white] (3.5,-2.5) rectangle ++(1,-1);
		\fill[white] (4.5,-2.5) rectangle ++(1,-1);
		\fill[green] (5.5,-2.5) rectangle ++(1,-1);
		\draw (0,-2.5) -- ++(3,0) -- ++(0,-1) -- ++(-3,0) -- ++(0,1) ++(1,0) -- ++(0,-1) -- ++(1,0) -- ++(0,1);
		\draw (-3.5,-2.5) -- ++(3,0) -- ++(0,-1) -- ++(-3,0) -- ++(0,1) ++(1,0) -- ++(0,-1) -- ++(1,0) -- ++(0,1);
		\draw (3.5,-2.5) -- ++(3,0) -- ++(0,-1) -- ++(-3,0) -- ++(0,1) ++(1,0) -- ++(0,-1) -- ++(1,0) -- ++(0,1);

		% connections
		\draw[line cap=round,black!50,shorten >=1.5cm, shorten <=1.5cm,->] (1.5,-3) -- (-2,-5.5);
		\draw[line cap=round,black!50,shorten >=1.5cm, shorten <=1.5cm,->] (1.5,-3) -- (5,-5.5);
		\draw[line cap=round,black!50,shorten >=0.75cm, shorten <=0.85cm,->] (5,-3) -- (5,-5.5);
		\draw[line cap=round,black!50,shorten >=1.5cm, shorten <=1.5cm,->] (5,-3) -- (1.5,-5.5);
		\draw[line cap=round,black!50,shorten >=0.75cm, shorten <=0.85cm,->] (-2,-3) -- (-2,-5.5);
		\draw[line cap=round,black!50,shorten >=1.5cm, shorten <=1.5cm,->] (-2,-3) -- (1.5,-5.5);

		% third line
		\fill[violet] (-3.5,-5) rectangle ++(1,-1);
		\fill[yellow] (-2.5,-5) rectangle ++(1,-1);
		\fill[white] (-1.5,-5) rectangle ++(1,-1);
		\fill[violet] (0,-5) rectangle ++(1,-1);
		\fill[white] (1,-5) rectangle ++(1,-1);
		\fill[green] (2,-5) rectangle ++(1,-1);
		\fill[white] (3.5,-5) rectangle ++(1,-1);
		\fill[yellow] (4.5,-5) rectangle ++(1,-1);
		\fill[green] (5.5,-5) rectangle ++(1,-1);
		\draw (0,-5) -- ++(3,0) -- ++(0,-1) -- ++(-3,0) -- ++(0,1) ++(1,0) -- ++(0,-1) -- ++(1,0) -- ++(0,1);
		\draw (-3.5,-5) -- ++(3,0) -- ++(0,-1) -- ++(-3,0) -- ++(0,1) ++(1,0) -- ++(0,-1) -- ++(1,0) -- ++(0,1);
		\draw (3.5,-5) -- ++(3,0) -- ++(0,-1) -- ++(-3,0) -- ++(0,1) ++(1,0) -- ++(0,-1) -- ++(1,0) -- ++(0,1);

		% connections
		\draw[line cap=round,black!50,shorten >=1.5cm, shorten <=1.5cm,->] (5,-5.5) -- (1.5,-8);
		\draw[line cap=round,black!50,shorten >=0.75cm, shorten <=0.85cm,->] (1.5,-5.5) -- (1.5,-8);
		\draw[line cap=round,black!50,shorten >=1.5cm, shorten <=1.5cm,->] (-2,-5.5) -- (1.5,-8);

		% fourth line
		\fill[violet] (0,-7.5) rectangle ++(1,-1);
		\fill[yellow] (1,-7.5) rectangle ++(1,-1);
		\fill[green] (2,-7.5) rectangle ++(1,-1);
		\draw (0,-7.5) -- ++(3,0) -- ++(0,-1) -- ++(-3,0) -- ++(0,1) ++(1,0) -- ++(0,-1) -- ++(1,0) -- ++(0,1);

		% annotations
		\node[black!50] at (-2,-0.8) (absent) {\commentsize\texthandwritten{absent}};
		\draw[black!50,commentarrow] (-0.6,-0.8) to[bend left=30] (0.5,-0.5);
		\node[black!50] at (5.2,-0.9) {\commentsize\texthandwritten{present}};
		\draw[black!50,commentarrow] (5.4,-1.2) to[bend left=30] (6,-3);
	\end{scope}
	% <<< Markov chain visualization

	% >>> MHN block
	\begin{scope}[shift={(8,36)}]
		\filldraw[blockcolors, line width=0.3cm]
		(-1,-10) ..controls ++(0,8) and ++(-15,-1)..
		(13,3) ..controls ++(18,1) and ++(-1,10)..
		(32,-10) ..controls ++(1.8,-18) and ++(10,1)..
		(21,-27.5) ..controls ++(-20,-2) and ++(0,-20)..
		cycle;
		\node[anchor=north west] at (0,0)
		{
			\parbox[t]{30cm}{
				\hspace{1cm}\headingsize\textbf{Mutual Hazard Networks}\normalsize\raisebox{0.4ex}{~\cite{Schill_2024}}
				\normalsize
				\begin{itemize}
					\item Cancer progresses by accumulating genetic events
					\item This progression can be modeled as a Markov chain with transition rates
					      \begin{align*}
						      \vec Q_{x^{+i},x}=\Theta_{ii}\prod_{x_j=1}^d\Theta_{ij}
					      \end{align*}
					\item Patient data of observed tumors define a probability distribution $\vec p_\mathcal D$
					      \begin{itemize}
						      \item[\followsarrow] Parameters $\Theta$ can be infered by comparing to the time-marginalized probability distribution
						            \begin{align*}
							            \vec p_\Theta=(\vec I-\vec Q)^{-1}\vec p_0
						            \end{align*}
						            via the log-likelihood (LL)
					      \end{itemize}
					\item Exact calculation of $\vec p_\Theta$ is limited to under 25 active events per patient due to runtime behavior
				\end{itemize}
			}
		};
		\node[black!50] at (15,-7.7) {\commentsize\parbox[t]{8cm}{
				\centering
				\texthandwritten{base rate\\of event i}
			}
		};
		\draw[commentarrow,black!50] (15,-8.6) to[bend right=20] (15.4,-9.5);
		\node[black!50] at (23,-8) {\commentsize\parbox[t]{8cm}{
				\centering
				\texthandwritten{influence of\\event j on event i}
			}
		};
		\draw[commentarrow,black!50] (21.05,-8.85) to[bend left=30] (20.35,-9.5);
		\node[black!50] at (26.2,-19.65) {\commentsize\parbox[t]{10cm}{
				\texthandwritten{initial distribution contains only healthy patients}
			}
		};
		\draw[commentarrow,black!50] (21,-19.2) to[bend right=30] (20.3,-19.7);
		\node[black!50] at (11,-11.5) {\commentsize\texthandwritten{rate to acquire event i}};
		\draw[commentarrow,black!50] (9.5,-11) to[bend left=40] (10.5,-10);
	\end{scope}
	% <<< MHN block

	% >>> Clustering example
	\begin{scope}[shift={(26,0)},font=\refsize]
		% Theta matrix
		\begin{scope}[shift={(-2.5,0)}]
			\node at (-5,5.5) {\texthandwritten{$\Theta$ matrix}};
			\node at (-8,4) {A};
			\node at (-6.5,4) {B};
			\node at (-5,4) {C};
			\node at (-3.5,4) {D};
			\node at (-2,4) {E};
			\node at (-0.5,4) {F};

			\node at (-9.5,2.5) {A};
			\node[gray] at (-8,2.5) {2};
			\node[red] at (-6.5,2.5) {.3};
			\node[blue] at (-5,2.5) {4};
			\node[gray] at (-3.5,2.5) {1};
			\node[gray] at (-2,2.5) {1};
			\node[blue] at (-0.5,2.5) {1.5};

			\node at (-9.5,1) {B};
			\node[red] at (-8,1) {.3};
			\node[gray] at (-6.5,1) {1};
			\node[red] at (-5,1) {.5};
			\node[gray] at (-3.5,1) {1};
			\node[gray] at (-2,1) {1};
			\node[gray] at (-0.5,1) {1};

			\node at (-9.5,-0.5) {C};
			\node[blue] at (-8,-0.5) {5};
			\node[blue] at (-6.5,-0.5) {2};
			\node[gray] at (-5,-0.5) {1};
			\node[gray] at (-3.5,-0.5) {1};
			\node[red] at (-2,-0.5) {.4};
			\node[gray] at (-0.5,-0.5) {1};

			\node at (-9.5,-2) {D};
			\node[gray] at (-8,-2) {1};
			\node[gray] at (-6.5,-2) {1};
			\node[gray] at (-5,-2) {1};
			\node[gray] at (-3.5,-2) {1.5};
			\node[blue] at (-2,-2) {4};
			\node[red] at (-0.5,-2) {.2};

			\node at (-9.5,-3.5) {E};
			\node[gray] at (-8,-3.5) {1};
			\node[gray] at (-6.5,-3.5) {1};
			\node[red] at (-5,-3.5) {.5};
			\node[blue] at (-3.5,-3.5) {3};
			\node[gray] at (-2,-3.5) {2};
			\node[red] at (-0.5,-3.5) {.1};

			\node at (-9.5,-5) {F};
			\node[blue] at (-8,-5) {3};
			\node[gray] at (-6.5,-5) {1};
			\node[gray] at (-5,-5) {1};
			\node[blue] at (-3.5,-5) {2};
			\node[red] at (-2,-5) {.2};
			\node[gray] at (-0.5,-5) {1};
		\end{scope}

		\node at (1.25,0) {\scalebox{0.7}{\tikz{\straightarrow(0,0)[cyan!50,rotate=0]}}};
		\node at (1.25,1.5) {\commentsize$\max(|\log\Theta_{ij}|,|\log\Theta_{ji}|)$};

		%Distance matrix
		\begin{scope}[shift={(3,1.5)}]
			\node at (7,5.5) {\texthandwritten{Distance matrix}};
			\node at (4,4) {A};
			\node at (5.5,4) {B};
			\node at (7,4) {C};
			\node at (8.5,4) {D};
			\node at (10,4) {E};
			\node at (11.5,4) {F};

			\node at (2.5,2.5) {A};
			\node[gray] at (4,2.5) {$\infty$};
			\node at (5.5,2.5) {.8};
			\node at (7,2.5) {.6};
			\node[gray] at (8.5,2.5) {$\infty$};
			\node[gray] at (10,2.5) {$\infty$};
			\node at (11.5,2.5) {.9};

			\node at (2.5,1) {B};
			\node at (4,1) {.8};
			\node[gray] at (5.5,1) {$\infty$};
			\node at (7,1) {1.4};
			\node[gray] at (8.5,1) {$\infty$};
			\node[gray] at (10,1) {$\infty$};
			\node[gray] at (11.5,1) {$\infty$};

			\node at (2.5,-0.5) {C};
			\node at (4,-0.5) {.6};
			\node at (5.5,-0.5) {1.4};
			\node[gray] at (7,-0.5) {$\infty$};
			\node[gray] at (8.5,-0.5) {$\infty$};
			\node at (10,-0.5) {1.1};
			\node[gray] at (11.5,-0.5) {$\infty$};

			\node at (2.5,-2) {D};
			\node[gray] at (4,-2) {$\infty$};
			\node[gray] at (5.5,-2) {$\infty$};
			\node[gray] at (7,-2) {$\infty$};
			\node[gray] at (8.5,-2) {$\infty$};
			\node at (10,-2) {.7};
			\node at (11.5,-2) {.6};

			\node at (2.5,-3.5) {E};
			\node[gray] at (4,-3.5) {$\infty$};
			\node[gray] at (5.5,-3.5) {$\infty$};
			\node at (7,-3.5) {1.1};
			\node at (8.5,-3.5) {.7};
			\node[gray] at (10,-3.5) {$\infty$};
			\node at (11.5,-3.5) {.4};

			\node at (2.5,-5) {F};
			\node at (4,-5) {.9};
			\node[gray] at (5.5,-5) {$\infty$};
			\node[gray] at (7,-5) {$\infty$};
			\node at (8.5,-5) {.6};
			\node at (10,-5) {.4};
			\node[gray] at (11.5,-5) {$\infty$};
		\end{scope}

		\node at (8,-6.5) {\scalebox{0.65}{\tikz{\straightarrow(0,0)[cyan!50,rotate=270]}}};

		%Clustering
		\node at (8,-9) {\texthandwritten{Clustering algorithm}};
		\begin{scope}[shift={(4,-14)}]
			\node[circle, draw=green, fill=green!20, inner sep=.4ex] at (2.3,1.1) (A) {\refsize A};
			\node[circle, draw=green, fill=green!20, inner sep=.4ex] at (0,3.3) (B) {\refsize B};
			\node[circle, draw=green, fill=green!20, inner sep=.4ex] at (-2.5,1.3) (C) {\refsize C};
			\node[circle, draw=green, fill=green!20, inner sep=.4ex] at (-0.5,-3.2) (D) {\refsize D};
			\node[circle, draw=green, fill=green!20, inner sep=.4ex] at (-2.2,-1) (E) {\refsize E};
			\node[circle, draw=green, fill=green!20, inner sep=.4ex] at (2,-1.2) (F) {\refsize F};
			\draw[red, -|, bend right=15, line width=3*1.2pt] (A) to (B);
			\draw[red, -|, bend right=15, line width=3*1.2pt] (B) to (A);
			\draw[blue,->, bend right=15, line width=3*1.39pt] (C) to (A);
			\draw[blue,->, bend right=15, line width=3*1.61pt] (A) to (C);
			\draw[red, -|, bend right=15, line width=3*0.69pt] (C) to (B);
			\draw[blue,->, bend right=15, line width=3*0.69pt] (B) to (C);
			\draw[blue,->, bend right=15, line width=3*0.41pt] (F) to (A);
			\draw[blue,->, bend right=15, line width=3*1.1pt] (A) to (F);
			\draw[red, -|, bend right=15, line width=3*0.92pt] (E) to (C);
			\draw[red, -|, bend right=15, line width=3*0.69pt] (C) to (E);
			\draw[blue,->, bend right=15, line width=3*1.39pt] (E) to (D);
			\draw[blue,->, bend right=15, line width=3*1.1pt] (D) to (E);
			\draw[red, -|, bend right=15, line width=3*1.61pt] (F) to (D);
			\draw[blue,->, bend right=15, line width=3*0.69pt] (D) to (F);
			\draw[red, -|, bend right=15, line width=3*2.3pt] (F) to (E);
			\draw[red, -|, bend right=15, line width=3*1.61pt] (E) to (F);
		\end{scope}
		\node at (8,-11) {\scalebox{0.35}{\tikz{\curvedarrow(0,0)[violet!50,rotate=-15]}}};
		\node[black!50, rotate=5] at (7.5,-10.2) {\commentsize\texthandwritten{1. step}};
		\begin{scope}[shift={(12,-14.5)}]
			\filldraw[draw=yellow,fill=yellow!50, line width=0.7mm]
			(-3.6,1.7) ..controls ++(75:-2.5) and ++(100:-2)..
			(3.4,1.2) ..controls ++(100:1.5) and ++(0:2)..
			(0,1.8) ..controls ++(0:-2) and ++(75:1.5)..
			cycle;

			\node[circle, draw=green, fill=green!20, inner sep=.4ex] at (2.3,1.1) (A) {\refsize A};
			\node[circle, draw=green, fill=green!20, inner sep=.4ex] at (0,3.3) (B) {\refsize B};
			\node[circle, draw=green, fill=green!20, inner sep=.4ex] at (-2.5,1.3) (C) {\refsize C};
			\node[circle, draw=green, fill=green!20, inner sep=.4ex] at (-0.5,-3.2) (D) {\refsize D};
			\node[circle, draw=green, fill=green!20, inner sep=.4ex] at (-2.2,-1) (E) {\refsize E};
			\node[circle, draw=green, fill=green!20, inner sep=.4ex] at (2,-1.2) (F) {\refsize F};
			\draw[red, -|, bend right=15, line width=3*1.2pt] (A) to (B);
			\draw[red, -|, bend right=15, line width=3*1.2pt] (B) to (A);
			\draw[blue,->, bend right=15, line width=3*1.39pt] (C) to (A);
			\draw[blue,->, bend right=15, line width=3*1.61pt] (A) to (C);
			\draw[red, -|, bend right=15, line width=3*0.69pt] (C) to (B);
			\draw[blue,->, bend right=15, line width=3*0.69pt] (B) to (C);
			\draw[blue,->, bend right=15, line width=3*0.41pt] (F) to (A);
			\draw[blue,->, bend right=15, line width=3*1.1pt] (A) to (F);
			\draw[red, -|, bend right=15, line width=3*0.92pt] (E) to (C);
			\draw[red, -|, bend right=15, line width=3*0.69pt] (C) to (E);
			\draw[blue,->, bend right=15, line width=3*1.39pt] (E) to (D);
			\draw[blue,->, bend right=15, line width=3*1.1pt] (D) to (E);
			\draw[red, -|, bend right=15, line width=3*1.61pt] (F) to (D);
			\draw[blue,->, bend right=15, line width=3*0.69pt] (D) to (F);
			\draw[red, -|, bend right=15, line width=3*2.3pt] (F) to (E);
			\draw[red, -|, bend right=15, line width=3*1.61pt] (E) to (F);
		\end{scope}
		\node at (14,-18.5) {\scalebox{0.35}{\tikz{\curvedarrow(0,0)[violet!50,rotate=-110]}}};
		\node[black!50, rotate=-80] at (14.9,-18.5) {\commentsize\texthandwritten{2. step}};
		\begin{scope}[shift={(12.1,-23.5)}]
			\filldraw[draw=yellow,fill=yellow!50, line width=0.7mm]
			(-3.6,1.7) ..controls ++(75:-2.5) and ++(100:-2)..
			(3.4,1.2) ..controls ++(100:1.5) and ++(0:2)..
			(0,1.8) ..controls ++(0:-2) and ++(75:1.5)..
			cycle;
			\filldraw[draw=cyan,fill=cyan!50, line width=0.7mm]
			(-3.4,-1) ..controls ++(85:-1.8) and ++(92:-1.5)..
			(3.2,-1.2) ..controls ++(92:1.5) and ++(0:1.5)..
			(0,-0.48) ..controls ++(0:-1.5) and ++(85:1.9)..
			cycle;

			\node[circle, draw=green, fill=green!20, inner sep=.4ex] at (2.3,1.1) (A) {\refsize A};
			\node[circle, draw=green, fill=green!20, inner sep=.4ex] at (0,3.3) (B) {\refsize B};
			\node[circle, draw=green, fill=green!20, inner sep=.4ex] at (-2.5,1.3) (C) {\refsize C};
			\node[circle, draw=green, fill=green!20, inner sep=.4ex] at (-0.5,-3.2) (D) {\refsize D};
			\node[circle, draw=green, fill=green!20, inner sep=.4ex] at (-2.2,-1) (E) {\refsize E};
			\node[circle, draw=green, fill=green!20, inner sep=.4ex] at (2,-1.2) (F) {\refsize F};
			\draw[red, -|, bend right=15, line width=3*1.2pt] (A) to (B);
			\draw[red, -|, bend right=15, line width=3*1.2pt] (B) to (A);
			\draw[blue,->, bend right=15, line width=3*1.39pt] (C) to (A);
			\draw[blue,->, bend right=15, line width=3*1.61pt] (A) to (C);
			\draw[red, -|, bend right=15, line width=3*0.69pt] (C) to (B);
			\draw[blue,->, bend right=15, line width=3*0.69pt] (B) to (C);
			\draw[blue,->, bend right=15, line width=3*0.41pt] (F) to (A);
			\draw[blue,->, bend right=15, line width=3*1.1pt] (A) to (F);
			\draw[red, -|, bend right=15, line width=3*0.92pt] (E) to (C);
			\draw[red, -|, bend right=15, line width=3*0.69pt] (C) to (E);
			\draw[blue,->, bend right=15, line width=3*1.39pt] (E) to (D);
			\draw[blue,->, bend right=15, line width=3*1.1pt] (D) to (E);
			\draw[red, -|, bend right=15, line width=3*1.61pt] (F) to (D);
			\draw[blue,->, bend right=15, line width=3*0.69pt] (D) to (F);
			\draw[red, -|, bend right=15, line width=3*2.3pt] (F) to (E);
			\draw[red, -|, bend right=15, line width=3*1.61pt] (E) to (F);
		\end{scope}
		\node at (8.3,-20.2) {\scalebox{0.35}{\tikz{\curvedarrowinv(0,0)[violet!50,rotate=200]}}};
		\node[black!50, rotate=-8] at (8.8,-19.5) {\commentsize\texthandwritten{3. step}};
		\begin{scope}[shift={(4.2,-23)}]
			\filldraw[draw=yellow,fill=yellow!50, line width=0.7mm]
			(-3.6,1.7) ..controls ++(75:-2.5) and ++(100:-2)..
			(3.4,1.2) ..controls ++(100:1.5) and ++(0:2)..
			(0,1.8) ..controls ++(0:-2) and ++(75:1.5)..
			cycle;
			\filldraw[draw=cyan,fill=cyan!50, line width=0.7mm]
			(-3.4,-1) ..controls ++(85:-1.8) and ++(355:-1)..
			(-0.5,-4.2) ..controls ++(355:1.4) and ++(92:-1.5)..
			(3.2,-1.2) ..controls ++(92:1.5) and ++(0:1.5)..
			(0,-0.48) ..controls ++(0:-1.5) and ++(85:1.9)..
			cycle;

			\node[circle, draw=green, fill=green!20, inner sep=.4ex] at (2.3,1.1) (A) {\refsize A};
			\node[circle, draw=green, fill=green!20, inner sep=.4ex] at (0,3.3) (B) {\refsize B};
			\node[circle, draw=green, fill=green!20, inner sep=.4ex] at (-2.5,1.3) (C) {\refsize C};
			\node[circle, draw=green, fill=green!20, inner sep=.4ex] at (-0.5,-3.2) (D) {\refsize D};
			\node[circle, draw=green, fill=green!20, inner sep=.4ex] at (-2.2,-1) (E) {\refsize E};
			\node[circle, draw=green, fill=green!20, inner sep=.4ex] at (2,-1.2) (F) {\refsize F};
			\draw[red, -|, bend right=15, line width=3*1.2pt] (A) to (B);
			\draw[red, -|, bend right=15, line width=3*1.2pt] (B) to (A);
			\draw[blue,->, bend right=15, line width=3*1.39pt] (C) to (A);
			\draw[blue,->, bend right=15, line width=3*1.61pt] (A) to (C);
			\draw[red, -|, bend right=15, line width=3*0.69pt] (C) to (B);
			\draw[blue,->, bend right=15, line width=3*0.69pt] (B) to (C);
			\draw[blue,->, bend right=15, line width=3*0.41pt] (F) to (A);
			\draw[blue,->, bend right=15, line width=3*1.1pt] (A) to (F);
			\draw[red, -|, bend right=15, line width=3*0.92pt] (E) to (C);
			\draw[red, -|, bend right=15, line width=3*0.69pt] (C) to (E);
			\draw[blue,->, bend right=15, line width=3*1.39pt] (E) to (D);
			\draw[blue,->, bend right=15, line width=3*1.1pt] (D) to (E);
			\draw[red, -|, bend right=15, line width=3*1.61pt] (F) to (D);
			\draw[blue,->, bend right=15, line width=3*0.69pt] (D) to (F);
			\draw[red, -|, bend right=15, line width=3*2.3pt] (F) to (E);
			\draw[red, -|, bend right=15, line width=3*1.61pt] (E) to (F);
		\end{scope}
		\node at (3.2,-28.8) {\scalebox{0.35}{\tikz{\curvedarrowinv(0,0)[violet!50,rotate=-50]}}};
		\node[black!50, rotate=-70] at (2.5,-29) {\commentsize\texthandwritten{4. step}};
		\begin{scope}[shift={(8,-31)}]
			\filldraw[draw=yellow,fill=yellow!50, line width=0.7mm]
			(-3.6,1.7) ..controls ++(75:-2.5) and ++(100:-2)..
			(3.4,1.2) ..controls ++(100:1.5) and ++(0:2)..
			(0,4.3) ..controls ++(0:-2) and ++(75:1.5)..
			cycle;
			\filldraw[draw=cyan,fill=cyan!50, line width=0.7mm]
			(-3.4,-1) ..controls ++(85:-1.8) and ++(355:-1)..
			(-0.5,-4.2) ..controls ++(355:1.4) and ++(92:-1.5)..
			(3.2,-1.2) ..controls ++(92:1.5) and ++(0:1.5)..
			(0,-0.48) ..controls ++(0:-1.5) and ++(85:1.9)..
			cycle;

			\node[circle, draw=green, fill=green!20, inner sep=.4ex] at (2.3,1.1) (A) {\refsize A};
			\node[circle, draw=green, fill=green!20, inner sep=.4ex] at (0,3.3) (B) {\refsize B};
			\node[circle, draw=green, fill=green!20, inner sep=.4ex] at (-2.5,1.3) (C) {\refsize C};
			\node[circle, draw=green, fill=green!20, inner sep=.4ex] at (-0.5,-3.2) (D) {\refsize D};
			\node[circle, draw=green, fill=green!20, inner sep=.4ex] at (-2.2,-1) (E) {\refsize E};
			\node[circle, draw=green, fill=green!20, inner sep=.4ex] at (2,-1.2) (F) {\refsize F};
			\draw[red, -|, bend right=15, line width=3*1.2pt] (A) to (B);
			\draw[red, -|, bend right=15, line width=3*1.2pt] (B) to (A);
			\draw[blue,->, bend right=15, line width=3*1.39pt] (C) to (A);
			\draw[blue,->, bend right=15, line width=3*1.61pt] (A) to (C);
			\draw[red, -|, bend right=15, line width=3*0.69pt] (C) to (B);
			\draw[blue,->, bend right=15, line width=3*0.69pt] (B) to (C);
			\draw[blue,->, bend right=15, line width=3*0.41pt] (F) to (A);
			\draw[blue,->, bend right=15, line width=3*1.1pt] (A) to (F);
			\draw[red, -|, bend right=15, line width=3*0.92pt] (E) to (C);
			\draw[red, -|, bend right=15, line width=3*0.69pt] (C) to (E);
			\draw[blue,->, bend right=15, line width=3*1.39pt] (E) to (D);
			\draw[blue,->, bend right=15, line width=3*1.1pt] (D) to (E);
			\draw[red, -|, bend right=15, line width=3*1.61pt] (F) to (D);
			\draw[blue,->, bend right=15, line width=3*0.69pt] (D) to (F);
			\draw[red, -|, bend right=15, line width=3*2.3pt] (F) to (E);
			\draw[red, -|, bend right=15, line width=3*1.61pt] (E) to (F);
		\end{scope}
		\node[black!50, rotate=20] at (11.5,-34.5) {\commentsize\texthandwritten{\parbox[t]{5cm}{\centering obtained clustering}}};
	\end{scope}
	% <<< Clustering example

	% >>> Clustering block
	\begin{scope}[shift={(-16,9)}]
		\filldraw[blockcolors, line width=0.3cm]
		(6,2) ..controls ++(5:5) and ++(110:10)..
		(28,-6) ..controls ++(110:-10) and ++(0:10)..
		(14,-18) ..controls ++(0:-15) and ++(85:-10)..
		(-2,-7) ..controls ++(85:8) and ++(5:-4)..
		cycle;
		\node[anchor=north west] at (0,0)
		{
			\parbox[t]{0.3\wl}{
				\hspace{1cm}\headingsize\textbf{Clustering}
				\normalsize
				\begin{itemize}
                    \item Cancer progression is widely assumed to be weakly modular~\cite{Iranzo_2022}
					      \begin{itemize}
						      \item[\followsarrow] Perform calculations on smaller clusters and combine results in the end
					      \end{itemize}
					\item To estimate $\Theta_{ij}$, we need a cluster with $<\!25$ events, containing both $i$ and $j$
					\item We use hierarchical clustering~\cite{Rokach_2005} to obtain possibly overlapping clusters
				\end{itemize}
			}
		};
	\end{scope}
	% <<< Clustering block

	% >>> Validation block
	\begin{scope}[shift={(-28,-31)}]

		\fill[gray!50, draw=gray!50!black, line width=0.3cm, rounded corners=2.3cm] (-1.5,0.3) rectangle ++(35,-20);

		\node[anchor=north west] at (16,-8.7) {\includegraphics{graphics/theta_validation/theta_validation.pdf}};
		\node[anchor=north west] at (-0.5,-8.7) {\includegraphics{graphics/gradient_validation/gradient_validation.pdf}};

		\node[anchor=north west] at (0,0)
		{
			\parbox[t]{32cm}{
				\smallheadingsize\textbtt{Validation}\\
				\texttt{\refsize
					To validate our method using artificial datasets and MHNs with 80 events, we check:
					\begin{enumerate}
						\item Gradient approximation accuracy
						\item Accuracy of learned MHNs
					\end{enumerate}}
			}
		};
	\end{scope}
	% <<< Validation block

	% >>> biological example
	\begin{scope}[shift={(5,-10)},font=\commentsize]
		\node[anchor=north] at (0,0)
		{
			\parbox[t]{35cm}{\centering
				\smallheadingsize\texthandwritten{biological results}\\
				\refsize
				MSK-CHORD~\cite{Jee_2024} data of 5907 LUADs, trained on 125 genetic events\\
				strongest 30 connections shown
			}
		};
		\begin{scope}[shift={(4,-17)},line cap=round,
				event/.style={fill=green!20, draw=green, rectangle, inner sep=0.4em, rectangle, line width=1mm, rounded corners=2mm},
				promoting/.style={blue,->, text=blue!50!black, rounded corners=1mm},
				inhibiting/.style={red,-|,text=red!50!black, rounded corners=1mm},
				effect/.style={inner sep=0.2ex}
			]
			\def\s{1.5}

			\node[event] at (0,0) (EGFR) {EGFR};
			\node[event] at (-4.7,3) (MYC/8q) {\parbox[t]{2.65cm}{\centering MYC/8q\\Amp}};
			\node[event] at (0,2.7) (TP53/17p) {\parbox[t]{2.83cm}{\centering TP53/17p\\Del}};
			\node[event] at (0,5.5) (EPHA7/6q) {\parbox[t]{3.35cm}{\centering EPHA7/6q\\Del}};
			\node[event] at (0,-3) (NKX2-1/14q) {\parbox[t]{3.7cm}{\centering NKX2-1/14q\\Amp}};
			\node[event] at (-6,-0.25) (GNAS/20q) {\parbox[t]{3.55cm}{\centering GNAS/20q\\Amp}};
			\node[event] at (-11,-0.25) (CCND1/11q) {\parbox[t]{3.5cm}{\centering CCND1/11q\\Amp}};
			\node[event] at (0, 9) (CDKN2A/9p) {\parbox[t]{4.05cm}{\centering CDKN2A/9p\\Del}};
			\node[event] at (-6, 9) (BCL2/18q) {\parbox[t]{3.7cm}{\centering BCL2/18q\\Del}};
			\node[event] at (6,9) (B2M/15q) {\parbox[t]{2.8cm}{\centering B2M/15q\\Del}};
			\node[event] at (6,5.5) (TERT/5p) {\parbox[t]{2.8cm}{\centering TERT/5p\\Amp}};
			\node[event] at (-11.5,3) (RMI2/16p) {\parbox[t]{3cm}{\centering RMI2/16p\\Amp}};
			\node[event] at (-13,9.3) (ARID5B/10q) {\parbox[t]{3.85cm}{\centering ARID5B/10q\\Del}};
			\node[event] at (-13,6.3) (CXCR4/2q) {\parbox[t]{3.35cm}{\centering CXCR4/2q\\Del}};
			\node[event] at (8,2.5) (KRAS) {KRAS};
			\node[event] at (8,0) (EGFR/17p) {\parbox[t]{3.15cm}{\centering EGFR/17p\\Amp}};
			\node[event] at (4.5,-3) (KEAP1) {KEAP1};
			\node[event] at (9,-3) (STK11/19p) {\parbox[t]{3.06cm}{\centering STK11/19p\\Del}};
			\node[event] at (13.5,-3) (STK11) {STK11};

			\draw[inhibiting, line width=2.62785962052687*\s] ($(EGFR.south)+(0,-0.1)$) -- ($(NKX2-1/14q.north)+(0,0.1)$);
			\draw[inhibiting, line width=2.090191137159693*\s] ($(TP53/17p.west)+(-0.1,0.15)$) -- ($(MYC/8q.east)+(0.1,-0.15)$);
			\draw[promoting, line width=2.6329435007018858*\s] ($(EGFR.west)+(-0.1,0.4)$) -| ($(MYC/8q.south east)+(-0.2,-0.1)$);
			\draw[promoting, line width=2.5837968354725245*\s] ($(MYC/8q.south)+(-0.7,-0.1)$) -- ($(GNAS/20q.north)+(0.6,0.1)$);
			\draw[inhibiting, line width=2.7770633563697484*\s] ($(EGFR.west)+(-0.1,0)$) -- ($(GNAS/20q.east)+(0.1,0.25)$);
			\draw[inhibiting, line width=3.4485835476815803*\s] ($(EPHA7/6q.north)+(-0.2,0.1)$) -- ($(CDKN2A/9p.south)+(-0.2,-0.1)$);
			\draw[promoting, line width=2.346640143756334*\s] ($(CDKN2A/9p.south)+(0.2,-0.1)$) -- ($(EPHA7/6q.north)+(0.2,0.1)$);
			\draw[inhibiting, line width=2.8634787659839303*\s] ($(EPHA7/6q.west)+(-0.1,0)$) -| ($(MYC/8q.north east)+(-0.2,0.1)$);
			\draw[promoting, line width=2.436940964753187*\s] ($(CDKN2A/9p.west)+(-0.1,0.2)$) -- ($(BCL2/18q.east)+(0.1,0.2)$);
			\draw[inhibiting, line width=2.384520201540039*\s] ($(BCL2/18q.east)+(0.1,-0.2)$) -- ($(CDKN2A/9p.west)+(-0.1,-0.2)$);
			\draw[promoting, line width=2.227204778748276*\s] ($(CDKN2A/9p.east)+(0.1,-0.2)$) -- ($(B2M/15q.west)+(-0.1,-0.2)$);
			\draw[inhibiting, line width=2.1935608542062455*\s] ($(B2M/15q.west)+(-0.1,0.2)$) -- ($(CDKN2A/9p.east)+(0.1,0.2)$);
			\draw[inhibiting, line width=2.83413942169526*\s] ($(BCL2/18q.south)+(0.7,-0.1)$) -- ($(MYC/8q.north)+(-0.6,0.1)$);
			\draw[inhibiting, line width=2.140461516655592*\s] ($(B2M/15q.south)+(0,-0.1)$) -- ($(TERT/5p.north)+(0,0.1)$);
			\draw[inhibiting, line width=2.545765941307774*\s] ($(BCL2/18q.south east)+(-0.2,-0.1)$) |- (-0.4,7.25) (0.4,7.25) -| ($(TERT/5p.north west)+(0.2,0.1)$);
			\draw[promoting, line width=2.177737488174746*\s] ($(EGFR.north east)+(-0.2,0.1)$) -- ++(0,0.3) -| ($(TERT/5p.south west)+(0.6,-0.1)$);
			\draw[inhibiting, line width=2.818964791738894*\s] ($(TP53/17p.east)+(0.1,0)$) -| ($(TERT/5p.south west)+(0.2,-0.1)$);
			\draw[inhibiting, line width=2.394091944573555*\s] ($(EPHA7/6q.east)+(0.1,0)$) -- ($(TERT/5p.west)+(-0.1,0)$);
			\draw[promoting, line width=2.3620930223508565*\s] ($(TERT/5p.south west)+(0.95,-0.1)$) -- ++(0,-3.35) ++(0,-1) |- (3,-1.2) -| ($(NKX2-1/14q.north east)+(-0.2,0.1)$);
			\draw[promoting, line width=2.3735738537723017*\s] ($(ARID5B/10q.south)+(0,-0.1)$) -- ($(CXCR4/2q.north)+(0,0.1)$);
			\draw[promoting, line width=2.1938684346702697*\s] ($(MYC/8q.west)+(-0.1,0)$) -- ($(RMI2/16p.east)+(0.1,0)$);
			\draw[promoting, line width=2.4937024529361387*\s] ($(MYC/8q.south west)+(-0.1,0.2)$) -| ($(CCND1/11q.north east)+(-0.2,0.1)$);
			\draw[inhibiting, line width=2.1889670936302164*\s] ($(EGFR.west)+(-0.1,-0.4)$) -| ($(GNAS/20q.south east)+(0.5,-0.8)$) -| ($(CCND1/11q.south east)+(-0.2,-0.1)$);
			\draw[promoting, line width=2.316964127961973*\s] ($(EGFR.east)+(0.1,0)$) -- ($(EGFR/17p.west)+(-0.1,0)$);
			\draw[inhibiting, line width=3.038326507008012*\s] ($(KRAS.west)+(-0.1,0.2)$) -| ($(EGFR/17p.north west)+(-0.6,0)$) |- ($(EGFR.north east)+(0.1,-0.2)$);
			\draw[inhibiting, line width=2.0847404963047738*\s] ($(EGFR.north east)+(0.1,-0.5)$) -| ($(EGFR/17p.north west)+(-0.3,0)$) |- ($(KRAS.west)+(-0.1,-0.2)$);
			\draw[inhibiting, line width=2.092679620228705*\s] ($(EGFR.south east)+(0.1,0.2)$) -| (4.5,-1) ++ (0,-0.4) -- ($(KEAP1.north)+(0,0.1)$);
			\draw[promoting, line width=2.5137728556255863*\s] ($(STK11/19p.west)+(-0.1,0)$) -- ($(KEAP1.east)+(0.1,0)$);
			\draw[promoting, line width=2.276630886867481*\s] ($(STK11/19p.east)+(0.1,0)$) -- ($(STK11.west)+(-0.1,0)$);
			\draw[promoting, line width=2.4411675614605217*\s] ($(MYC/8q.south east)+(0.1,0.2)$) -| ($(NKX2-1/14q.north west)+(0.2,2.4)$) ++(0,-1.4) -- ($(NKX2-1/14q.north west)+(0.2,0.1)$);
			\draw[inhibiting, line width=2.7100698079181345*\s] ($(TP53/17p.north east)+(0.1,-0.2)$) -| ($(CDKN2A/9p.south east)+(-0.2,-2.6)$) ++(0,0.6) -- ++(0,1.2) ++(0,0.5) -- ($(CDKN2A/9p.south east)+(-0.2,-0.1)$);

			\node[black!50, rotate=-30] at (10.5,10.5) {\commentsize\texthandwritten{\parbox[t]{8cm}{\centering arrow width denotes\\effect strength}}};
		\end{scope}
	\end{scope}
	% <<< Weakly modular example

	% >>> Optimization process
	\begin{scope}[shift={(-40,-7)}]
		\filldraw[blockcolors, line width=0.3cm]
		(-1,-8) ..controls ++(85:10) and ++(0:-5)..
		(7,2) ..controls ++(0:10) and ++(100:15)..
		(29.5,-13) ..controls ++(100:-10) and ++(5:5)..
		(11,-22.5) ..controls ++(5:-14.5) and (85:-16)..
		cycle;
		\node[anchor=north west] at (0,0)
		{
			\parbox[t]{28cm}{
				\hspace{1cm}\headingsize\textbf{Learning Process}
				\normalsize
				\begin{itemize}
					\item To infer MHNs from patient data, we can\\utilize the structure found by our clustering:
					      \begin{enumerate}
						      \item Start at independence model, i.e.\ $\Theta_{i\ne j}=1$
						      \item For every parameter $\Theta_{ij}$: Get gradients of\\the LL score by considering a cluster\\containing events $i$ and $j$
						      \item Get new parameters $\Theta$ through one optimizer step
					      \end{enumerate}
					\item The clusters used to calculate gradients adapt throughout the optimization process to\\fit the data
				\end{itemize}
			}
		};
	\end{scope}
	% <<< Optimization process

	% >>> Future Work
	\begin{scope}[shift={(10,-35)}]
		\filldraw[blockcolors, line width=0.3cm]
		(-1,-8) ..controls ++(92:10) and ++(-5:-8)..
		(8,1) ..controls ++(-5:15) and ++(105:8)..
		(30.5,-10) ..controls ++(105:-13) and ++(5:10)..
		(12,-23) ..controls ++(5:-12) and ++(92:-12)..
		cycle;
		\node[anchor=north west] at (0,0)
		{
			\parbox[t]{30cm}{
				\hspace{1cm}\headingsize\textbf{Next steps}
				\normalsize
				\begin{itemize}
					\item Investigate choice of distance measure\\analytically and define it to minimize\\$\frac{\|g_{\text{exact}}-g_{\text{approx}}\|}{\|g_{\text{exact}}\|}$
					\item Obtain an approximation of the score along\\with the gradient approximation
					\item Consider different clustering strategies
					      \begin{itemize}
						      \item[\followsarrow] Spectral clustering is of particular interest, as first tests showed promising results on graphs obtained from MHNs
					      \end{itemize}
					\item Investigate biological interpretability of clusters\\\hphantom{n}further
				\end{itemize}
			}
		};
	\end{scope}
	% <<< Future Work

	% >>> References
	\begin{scope}[shift={(-41,-49)}]
		\node[anchor=north west] at (0,0)
		{
			\parbox[t]{50cm}{
				\AtNextBibliography{\reffsize}
				\printbibliography[title=\normalsize References\vspace{-0.5cm}]
			}
		};
	\end{scope}
	% <<< References
\end{tikzpicture}
\end{document}
